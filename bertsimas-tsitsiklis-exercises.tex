% Document setup
\documentclass[article, a4paper, 11pt, oneside]{memoir}
\usepackage[utf8]{inputenc}
\usepackage[T1]{fontenc}
\usepackage[UKenglish]{babel}

% Document info
\newcommand\doctitle{Bertsimas, Tsitsiklis: \emph{Introduction to Linear Optimization}}
\newcommand\docauthor{Danny Nygård Hansen}

% Formatting and layout
\usepackage[autostyle]{csquotes}
\renewcommand{\mktextelp}{(\textellipsis\unkern)}
\usepackage[final]{microtype}
\usepackage{xcolor}
\definecolor{chaptercolor}{HTML}{6b6bb3}
\frenchspacing
\usepackage{latex-sty/sfarticlepagestyle}
\usepackage{latex-sty/sfarticlesectionstyle}

% Fonts
\usepackage{amssymb}
\usepackage[largesmallcaps,partialup]{kpfonts}
\DeclareSymbolFontAlphabet{\mathrm}{operators} % https://tex.stackexchange.com/questions/40874/kpfonts-siunitx-and-math-alphabets
\linespread{1.06}
% \let\mathfrak\undefined
% \usepackage{eufrak}
\DeclareMathAlphabet\mathfrak{U}{euf}{m}{n}
\SetMathAlphabet\mathfrak{bold}{U}{euf}{b}{n}
% https://tex.stackexchange.com/questions/13815/kpfonts-with-eufrak
\usepackage{inconsolata}

% Hyperlinks
\usepackage{hyperref}
\definecolor{linkcolor}{HTML}{4f4fa3}
\hypersetup{%
	pdftitle=\doctitle,
	pdfauthor=\docauthor,
	colorlinks,
	linkcolor=linkcolor,
	citecolor=linkcolor,
	urlcolor=linkcolor,
	bookmarksnumbered=true
}

% Equation numbering
\numberwithin{equation}{chapter}

% Footnotes
\footmarkstyle{\textsuperscript{#1}\hspace{0.25em}}

% Mathematics
\usepackage{latex-sty/basicmathcommands}
\usepackage{latex-sty/framedtheorems}
\usepackage{latex-sty/theoremrefs}
\usepackage{latex-sty/topologycommands}
\usepackage{tikz-cd}
\tikzcdset{arrow style=math font} % https://tex.stackexchange.com/questions/300352/equalities-look-broken-with-tikz-cd-and-math-font
\usetikzlibrary{babel}

% Lists
\usepackage{enumitem}
\setenumerate[0]{label=\normalfont(\alph*)}

% Bibliography
\usepackage[backend=biber, style=authoryear, maxcitenames=2, useprefix]{biblatex}
\addbibresource{references.bib}

% Title
\title{\doctitle}
\author{\docauthor}


%% Framed exercise environment

\mdfdefinestyle{swannexercise}{%
    skipabove=0.5em plus 0.4em minus 0.2em,
	skipbelow=0.5em plus 0.4em minus 0.2em,
	leftmargin=-5pt,
	rightmargin=-5pt,
	innerleftmargin=5pt,
	innerrightmargin=5pt,
	innertopmargin=5pt,
	innerbottommargin=4pt,
	linewidth=0pt,
	splittopskip=1.2em minus 0.2em,
	splitbottomskip=0.5em plus 0.2em minus 0.1em,
	backgroundcolor=backgroundcolor,
	frametitlebackgroundcolor=titlecolor,
	frametitlefont={\scshape},
    theoremseparator={},
    % theoremspace={},
	frametitleaboveskip=3pt,
	frametitlebelowskip=2pt
}

\mdtheorem[style=swannexercise]{exerciseframed}{Exercise}

\let\oldexerciseframed\exerciseframed
\renewcommand{\exerciseframed}{%
  \crefalias{theorem}{exerciseframed}%
  \oldexerciseframed}

\usepackage{listofitems}

\settocdepth{subsection}
\renewenvironment{exerciseframed}[1][]{%
    \setsepchar{.}%
    \readlist*\mylist{#1}%
    \def\smalllabel{\mylist[2].\mylist[3]}%
    \refstepcounter{exerciseframed}%
    % \addcontentsline{toc}{subsection}{Exercise \smalllabel}%
    \begin{exerciseframed*}[#1]%
    \label{ex:#1}%
}{%
    \end{exerciseframed*}%
}

% https://tex.stackexchange.com/a/23491/63353
\newcommand{\RNum}[1]{\uppercase\expandafter{\romannumeral #1\relax}}

\newcommand{\exref}[1]{%
    % \setsepchar{.}%
    % \readlist*\mylist{#1}%
    % \ifnum \arabic{chapter}=\mylist[1]
    %     \def\mylabel{\mylist[2].\mylist[3]}%
    % \else
    %     \def\mylabel{\RNum{\mylist[1]}.\mylist[2].\mylist[3]}%
    % \fi
    \hyperref[ex:#1]{Exercise~#1}%
}


\mdtheorem[style=swannexercise]{problemframed}{Problem}

\let\oldproblemframed\problemframed
\renewcommand{\problemframed}{%
  \crefalias{theorem}{problemframed}%
  \oldproblemframed}

\renewenvironment{problemframed}[1][]{%
    \setsepchar{.}%
    \readlist*\mylist{#1}%
    \def\smalllabel{\mylist[2].\mylist[3]}%
    \refstepcounter{problemframed}%
    % \addcontentsline{toc}{subsection}{Exercise \smalllabel}%
    \begin{problemframed*}[#1]%
    \label{prob:#1}%
}{%
    \end{problemframed*}%
}

\newcommand{\probref}[1]{%
    % \setsepchar{.}%
    % \readlist*\mylist{#1}%
    % \ifnum \arabic{chapter}=\mylist[1]
    %     \def\mylabel{\mylist[2].\mylist[3]}%
    % \else
    %     \def\mylabel{\RNum{\mylist[1]}.\mylist[2].\mylist[3]}%
    % \fi
    \hyperref[prob:#1]{Problem~#1}%
}


\theoremstyle{nonumberplain}
\theoremsymbol{\ensuremath{\square}}
\newtheorem{solution}{Solution}

\let\oldsolution\solution
\renewcommand{\solution}{%
  \crefalias{theorem}{solution}%
  \oldsolution}

\newcommand{\solutionlabelfont}[1]{{\normalfont\color{linkcolor}#1}}
\newlist{solutionsec}{enumerate}{1}
\setlist[solutionsec]{leftmargin=0pt, parsep=0pt, listparindent=\parindent, font=\solutionlabelfont, label=(\alph*), labelsep=0pt, labelwidth=20pt, itemindent=20pt, align=left, itemsep=10pt}

\newenvironment{displaytheorem}{%
	\begin{displayquote}\itshape%
}{%
	\end{displayquote}%
}


\newcommand{\calB}{\mathcal{B}}
\newcommand{\calV}{\mathcal{V}}
\newcommand{\calU}{\mathcal{U}}
\newcommand{\calF}{\mathcal{F}}
\newcommand{\calE}{\mathcal{E}}
\newcommand{\calT}{\mathcal{T}}
\newcommand{\calO}{\mathcal{O}}
\newcommand{\bbF}{\mathbb{F}}

% TODO move to .sty?
\renewcommand{\disunion}{\sqcup}

\let\oldcoprod\coprod
\renewcommand{\coprod}{\sqcup}
\newcommand{\bigcoprod}{\oldcoprod}


\begin{document}

\maketitle

\addtocounter{chapter}{1}
\chapter{The geometry of linear programming}

\newcommand{\keyword}[1]{{\bfseries \textit{#1}}}
\newcommand{\mat}{\vec}


\begin{remarkbreak}[Adjacent basic solutions and adjacent bases]
    Let $P = \set{\vec{x} \in \reals^n}{\mat{A}\vec{x} = \vec{b}, \vec{x} \geq \vec{0}}$ be a polyhedron in standard form.
    %
    \begin{displaytheorem}
        If $\vec{x}$ and $\vec{y}$ are adjacent basic solutions, then they can be obtained from two adjacent bases.
    \end{displaytheorem}
    %
    Assume that $x_i = 0$ for $i \neq B(1), \ldots, B(m)$. Each of the $m$ equality constraints are active at both $\vec{x}$ and $\vec{y}$, so there must be exactly one $k$ such that $y_{B(k)} \neq 0$. Then $\vec{x}$ can be obtained from the basis with basic indices $\{B(1), \ldots, B(m)\}$ and $\vec{y}$ from a basis with basic indices $\{B(1), \ldots, B(k-1), j, B(k+1), \ldots, B(m)\}$ where $j \neq B(k)$. Hence these two bases are adjacent as claimed.

    \begin{displaytheorem}
        If two adjacent bases lead to distinct basic solutions $\vec{x}$ and $\vec{y}$, then these are adjacent.
    \end{displaytheorem}
    %
    Similarly, there are (at least) $n-1$ constraints that are active at both $\vec{x}$ and $\vec{y}$ (the equality constraints along with all but one of the constraints given by the $B(k)$). Since $\vec{x} \neq \vec{y}$ there must be some $k$ such that $x_{B(k)} = 0$ and $y_{B(k)} \neq 0$, so there are exactly $n-1$ constraints that are active at both $\vec{x}$ and $\vec{y}$.
\end{remarkbreak}


\begin{exerciseframed}[2.5]
    (Extreme points of isomorphic polyhedra) A mapping $f$ is called \keyword{affine} if it is of the form $f(\vec{x}) = \mat{A}\vec{x} + \vec{b}$, where $\mat{A}$ is a matrix and $\vec{b}$ is a vector. Let $P$ and $Q$ be polyhedra in $\reals^n$ and $\reals^m$, respectively. We say that $P$ and $Q$ are \keyword{isomorphic} if there exist affine mappings $f \colon P \to Q$ and $g \colon Q \to P$ such that $g(f(\vec{x})) = \vec{x}$ for all $\vec{x} \in P$, and $f(g(\vec{y})) = \vec{y}$ for all $\vec{y} \in Q$. (Intuitively, isomorphic polyhedra have the same shape.)
    %
    \begin{enumerate}
        \item If $P$ and $Q$ are isomorphic, show that there exists a one-to-one correspondence between their extreme points. In particular, if $f$ and $g$ are as above, show that $\vec{x}$ is an extreme point of $P$ if and only if $f(\vec{x})$ is an extreme point of $Q$.
        
        \item (Introducing slack variables leads to an isomorphic polyhedron) Let $P = \set{\vec{x} \in \reals^n}{\mat{A}\vec{x} \geq \vec{b}, \vec{x} \geq \vec{0}}$, where $\mat{A}$ is a matrix of dimensions $k \times n$. Let $Q = \set{(\vec{x},\vec{z}) \in \reals^{n+k}}{\mat{A}\vec{x}-\vec{z} = \vec{b}, \vec{x} \geq \vec{0}, \vec{x} \geq \vec{0}}$. Show that $P$ and $Q$ are isomorphic.
    \end{enumerate}
\end{exerciseframed}

\begin{solution}
\begin{solutionsec}
    \item For $\vec{x}, \vec{y}, \vec{z} \in P$ with $\vec{x} = \lambda \vec{y} + (1-\lambda) \vec{z}$, then since $f$ is affine we have $f(\vec{x}) = \lambda f(\vec{y}) + (1-\lambda) f(\vec{z})$. Hence if $\vec{x}$ is \emph{not} an extreme point, then neither is $f(\vec{x})$. Since $f$ is bijective with an affine inverse, this proves the claim.

    \item Simply note that the map $f \colon P \to Q$ given by $f(\vec{x}) = (\vec{x}, \mat{A}\vec{x} - \vec{b})$ is an isomorphism.
\end{solutionsec}
\end{solution}


\begin{exerciseframed}[2.7]
    Suppose that $\set{\vec{x} \in \reals^n}{\vec{a}_i' \vec{x} \geq b_i, i = 1, \ldots, m}$ and $\set{\vec{x} \in \reals^n}{\vec{g}_i' \vec{x} \geq h_i, i = 1, \ldots, k}$ are representations of the same nonempty polyhedron $P$. Suppose that the vectors $\vec{a}_1, \ldots, \vec{a}_m$ span $\reals^n$. Show that the same must be true for the vectors $\vec{g}_1, \ldots, \vec{g}_k$.
\end{exerciseframed}

\begin{solution}
    If $\vec{a}_1, \ldots, \vec{a}_m$ span $\reals^n$, then they must contain a collection of $n$ linearly independent vectors. Then Theorem~2.6 implies that $P$ has an extreme point. But this is a geometric property, so it does not depend on the representation. The same theorem then implies that the collection $\vec{g}_1, \ldots, \vec{g}_k$ contains $n$ linearly independent vectors. But then they span $\reals^n$.
\end{solution}


\begin{exerciseframed}[2.12]
    Consider a nonempty polyhedron $P$ and suppose that for each variable $x_i$ we have either the constraint $x_i \geq 0$ or the contraint $x_i \geq 0$. Is it true that $P$ has at least one basic feasible solution?
\end{exerciseframed}

\begin{solution}
    Let $I$ be the set of indices $i$ such that we have the contraint $x_i \leq 0$, and let $\mat{A}$ be a diagonal matrix with a $1$ as the $i$th entry on the diagonal if $i \not\in I$, and $-1$ if $i \in I$. The map $\vec{x} \mapsto \mat{A}\vec{x}$ is then an isomorphism between $P$ and the polyhedron $Q$ which is defined by the same contraints as $P$, except that the contraints on the form $x_i \leq 0$ are replaced by contraints $x_i \geq 0$. In particular, $P$ and $Q$ have the same extreme points by [TODO ref] Exercise 2.4, hence the same basic feasible solutions by Theorem~2.3.

    Next introduce slack variables to $Q$, yielding a polyhedron $R$ in standard form, which is isomorphic to $Q$ by [TODO ref] Exercise~2.5. But since $R$ is nonempty (since it is isomorphic to $P$) it has a basic feasible solution by Corollary~2.2.
\end{solution}


TODO 2.14, 2.15


\begin{exerciseframed}[2.21]
    Suppose that Fourier--Motzkin elimination is used in the manner described at the end of Section~2.8 to find the optimal cost in a linear programming problem. Show how this approach can be augmented to obtain an optimal solution as well.
\end{exerciseframed}

\begin{solution}
    Let $P \subseteq \reals^n$ be a polyhedron, and assume that we wish to minimise the quantity $\vec{c}'\vec{x}$ subject to $\vec{x} \in P$. Define
    %
    \begin{equation*}
        P_n
            = \set{(x_0, \vec{x}) \in \reals^{n+1}}{x_0 = \vec{c}'\vec{x} \text{ and } \vec{x} \in P},
    \end{equation*}
    %
    which is clearly a polyhedron in $\reals^{n+1}$. To minimise $\vec{c}'\vec{x}$ with $\vec{x} \in P$ we thus find the minimal value of $x_0$ among elements $(x_0, \vec{x}) \in P_n$. Letting $P_{n-k} = \Pi_k(P_n)$ for $k=1, \ldots, n$ we have $Q \defn P_0 = \Pi_n(P_n) = \set{x_0 \in \reals}{\exists \vec{x} \in P \colon (x_0,\vec{x}) \in P_n}$, so this is equivalent to finding the minimum of $Q$. Given the minimum of $Q$ we wish to find a (not necessarily unique) $\vec{x} \in P$ such that $(\min Q, \vec{x}) \in P_n$.

    We do this by recursively finding a suitable value for the variable we eliminate in each step of the algorithm. In the notation of the Elimination algorithm on page 72, let $P \subseteq \reals^n$ be a polyhedron, and let $Q = \Pi_{n-1}(P)$ be the polyhedron obtained when eliminating the variable $x_n$. Let $\overline{\vec{x}}$ be an element of $Q$, i.e. a vector satisfying the inequalities
    %
    \begin{alignat*}{2}
        d_j + \vec{f}_j' \overline{\vec{x}}
            &\geq d_i + \vec{f}_i' \overline{\vec{x}}, \qquad
            && \text{if $a_{in} > 0$ and $a_{jn} < 0$,} \\
        0
            &\geq d_k + \vec{f}_k' \overline{\vec{x}}, \qquad
            && \text{if $a_{kn} = 0$,}
    \end{alignat*}
    %
    for $i,j,k = 1, \ldots, n$. Now choose any $x_n \in \reals$ that lies in all the intervals $[d_i + \vec{f}_i' \overline{\vec{x}}, d_j + \vec{f}_j' \overline{\vec{x}}]$. Then the vector $(\overline{\vec{x}}, x_n)$ satisfies the inequalities
    %
    \begin{alignat*}{2}
        x_n
            &\geq d_i + \vec{f}_i' \overline{\vec{x}}, \qquad
            && \text{if $a_{in} > 0$,} \\
        d_j + \vec{f}_j' \overline{\vec{x}}
            &\geq x_n, \qquad
            && \text{if $a_{jn} < 0$,} \\
        0
            &\geq d_k + \vec{f}_k' \overline{\vec{x}}, \qquad
            && \text{if $a_{kn} = 0$.}
    \end{alignat*}
    %
    Such an $x_n$ exists since these inequalities are a representation of $P$, and the former inequalities are a representation of $Q$, which is the projection of $P$ by Theorem~2.10. Thus $(\overline{\vec{x}}, x_n) \in P$ as desired.
\end{solution}


\begin{exerciseframed}[2.22]
    Let $P$ and $Q$ be polyhedra in $\reals^n$.
    %
    \begin{enumerate}
        \item Show that $P + Q$ is a polyhedron.

        \item Show that every extreme point of $P + Q$ is the sum of an extreme point of $P$ and an extreme point of $Q$.
    \end{enumerate}
\end{exerciseframed}

\begin{solution}
\begin{solutionsec}
    \item The Cartesian product $P \times Q$ is clearly a polyhedron. Its image under the linear map $(\vec{x}, \vec{y}) \mapsto \vec{x} + \vec{y}$ is exactly $P + Q$, and this is a polyhedron by Corollary~2.5.

    \item Let $\vec{z} \in P + Q$ and write $\vec{z} = \vec{x} + \vec{y}$ for $\vec{x} \in P$ and $\vec{y} \in Q$. Assume that $\vec{x}$ is not an extreme point in $P$, so that there exist $\vec{x}_1, \vec{x}_2 \in P \setminus \{\vec{x}\}$ and a $\lambda \in [0,1]$ such that $\vec{x} = \lambda \vec{x}_1 + (1-\lambda) \vec{x}_2$. Letting $\vec{z}_i \defn \vec{x}_i + \vec{y}$ we easily find that $\vec{z} = \lambda \vec{z}_1 + (1-\lambda) \vec{z}_2$, so $\vec{z}$ is not an extreme point.
\end{solutionsec}
\end{solution}


\end{document}